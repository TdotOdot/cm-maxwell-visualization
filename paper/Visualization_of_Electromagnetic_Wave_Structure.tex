\documentclass[11pt,a4paper]{article}
\usepackage[T1]{fontenc}
\usepackage[utf8]{inputenc}
\usepackage{amsmath,amssymb,amsfonts,mathrsfs}
\usepackage{geometry}
\geometry{margin=1in}
\usepackage{graphicx}
\usepackage{booktabs}
\usepackage{tikz}
\usetikzlibrary{arrows.meta, positioning, shapes.geometric}
\usepackage{caption}
\captionsetup[table]{skip=10pt}
\usepackage{hyperref}
\hypersetup{
    colorlinks=true,
    linkcolor=blue,
    citecolor=blue,
    urlcolor=blue,
    pdfstartview=FitH
}
\usepackage[style=numeric,sorting=none,doi=true,url=true,backend=biber]{biblatex}

\title{Visualization of Electromagnetic Wave Structure from the CM-Maxwell Unified Equation}
\author{T.O. (IASER Practitioner) \\[0.5em]
\small {AI Collaborators: Qwen (Alibaba Cloud), ChatGPT (OpenAI), Gemini (Google), Claude (Anthropic)}}
\date{January, 2026}

\begin{filecontents}{references.bib}
@article{to2026unified,
  author = {T.O.},
  title = {A Unified Maxwell Equation Derived from Cognitional Mechanics: A Practical Foundation for Engineering Electromagnetics},
  journal = {Zenodo},
  year = {2026},
  month = {January},
  doi = {10.5281/zenodo.18312668},
  url = {https://doi.org/10.5281/zenodo.18312668}
}

@article{to2026m3c,
  author = {T.O.},
  title = {M₃(ℂ) Necessity in Cognitional Mechanics: The Logical Foundation of Dimensional Structure},
  journal = {Zenodo},
  year = {2026},
  doi = {10.5281/zenodo.18285838},
  url = {https://doi.org/10.5281/zenodo.18285838}
}
\end{filecontents}

\addbibresource{references.bib}

\begin{document}

\maketitle

\begin{abstract}
We present an interactive visualization of electromagnetic wave structure derived from the Cognitional Mechanics (CM) unified Maxwell equation. The plane wave solution in vacuum demonstrates the orthogonal relationship between electric field $\mathbf{E}$, magnetic field $\mathbf{B}$, and Poynting vector $\mathbf{S}$, alongside their phase synchronization—properties that emerge naturally from the $\mathfrak{su}(3)$-valued field representation in CM theory. The visualization code is publicly available and serves both as pedagogical material and empirical validation of the CM-Maxwell formalism's geometric structure. This work supplements our main paper on the unified Maxwell equation~\cite{to2026unified}.
\end{abstract}

\section{Introduction}
\label{sec:intro}

The CM-Maxwell unified equation~\cite{to2026unified}
\begin{equation}
\left(\frac{1}{c}\frac{\partial}{\partial t} + \nabla \cdot \right) \mathbf{F} = \mathbf{J}
\label{eq:cm-maxwell}
\end{equation}
encodes all four classical Maxwell equations in a single operator acting on the complex electromagnetic matrix
\begin{equation}
\mathbf{F} = \begin{pmatrix}
0 & f_x & f_y \\
-f_x^* & 0 & f_z \\
-f_y^* & -f_z^* & 0
\end{pmatrix}, \quad f = \mathbf{E} + ic\mathbf{B}.
\label{eq:F-matrix}
\end{equation}

While the algebraic equivalence to standard Maxwell theory has been established, the geometric meaning of this $\mathfrak{su}(3)$ representation benefits from direct visualization. This paper presents a computational implementation that renders the electromagnetic wave solution in three-dimensional space, explicitly showing:

\begin{enumerate}
\item The mutual orthogonality $\mathbf{E} \perp \mathbf{B} \perp \mathbf{S}$
\item Phase synchronization between $\mathbf{E}$ and $\mathbf{B}$ fields
\item The directionality of energy flux via the Poynting vector $\mathbf{S} = (1/\mu_0)\mathbf{E} \times \mathbf{B}$
\end{enumerate}

\section{Plane Wave Solution}
\label{sec:solution}

In vacuum ($\rho = 0$, $\mathbf{J} = 0$), the CM-Maxwell equation reduces to the wave equation. We consider a circularly polarized plane wave propagating in the $+z$ direction:

\begin{align}
\mathbf{E}(z,t) &= E_0 \cos(kz - \omega t) \, \hat{\mathbf{x}} \label{eq:E}\\
\mathbf{B}(z,t) &= B_0 \cos(kz - \omega t) \, \hat{\mathbf{y}} \label{eq:B}\\
\mathbf{S}(z,t) &= \frac{1}{\mu_0} \mathbf{E} \times \mathbf{B} = \frac{E_0 B_0}{\mu_0} \cos^2(kz - \omega t) \, \hat{\mathbf{z}} \label{eq:S}
\end{align}

where $k = 2\pi/\lambda$, $\omega = ck$, and $c = 1/\sqrt{\epsilon_0\mu_0}$. For visualization purposes, we freeze time at $t=0$ and sample the fields at discrete $z$ positions.

\section{Visualization Implementation}
\label{sec:implementation}

\subsection{Geometric Representation}

At each spatial point $(0, 0, z_i)$ along the propagation axis, we construct three arrows representing $\mathbf{E}$, $\mathbf{B}$, and $\mathbf{S}$, all originating from the same point. This common-origin representation physically reflects that the Poynting vector is computed \emph{at the same spacetime point} as the fields:
\begin{equation}
\mathbf{S}(z,t) = \frac{1}{\mu_0} \mathbf{E}(z,t) \times \mathbf{B}(z,t).
\label{eq:poynting-local}
\end{equation}

The field trajectories are visualized as helices in $(x,y,z)$ space, with parametric equations:
\begin{align}
\mathbf{r}_E(z) &= (\cos(kz), \sin(kz), z) \label{eq:helix-E}\\
\mathbf{r}_B(z) &= (-\sin(kz), \cos(kz), z) \label{eq:helix-B}
\end{align}

\subsection{Technical Details}

The implementation uses Python with the Plotly library for interactive 3D rendering. Key design choices:

\begin{itemize}
\item \textbf{Arrow construction}: Manual mesh geometry (not built-in Cone objects) ensures view-independent visual consistency
\item \textbf{Color coding}: Red ($\mathbf{E}$), blue ($\mathbf{B}$), green ($\mathbf{S}$)
\item \textbf{Length scaling}: $\mathbf{E}$ and $\mathbf{B}$ arrows are scaled to $0.55 \times L$ relative to $\mathbf{S}$ arrow length $L$ for visual balance (this is purely aesthetic; physical field magnitudes follow Eqs.~\ref{eq:E}--\ref{eq:S})
\item \textbf{Camera position}: Set to $(x,y,z)_{\text{eye}} = (-1.8, -1.8, 1.5)$ to ensure the $+z$ propagation direction points away from the viewer
\end{itemize}

The complete source code is available at \url{https://github.com/TdotOdot/cm-maxwell-visualization} with MIT license.

\begin{figure}[htbp]
\centering
\includegraphics[width=0.9\textwidth]{fig1.png}
\caption{Electromagnetic wave structure visualization. Red arrows: electric field $\mathbf{E}$; blue arrows: magnetic field $\mathbf{B}$; green arrows: Poynting vector $\mathbf{S}$. Red and blue helices show continuous field trajectories. All three vectors originate from common points on the z-axis, demonstrating mutual orthogonality and phase synchronization. Interactive version available at \url{https://github.com/TdotOdot/cm-maxwell-visualization}.}
\label{fig:emwave}
\end{figure}

\section{Results and Verification}
\label{sec:results}

\subsection{Orthogonality}

Figure~\ref{fig:emwave} demonstrates the mutual perpendicularity of all three vector fields. At each sample point, the triplet $(\mathbf{E}, \mathbf{B}, \mathbf{S})$ forms a right-handed orthogonal basis, as required by Maxwell theory. This orthogonality is \emph{not imposed} in the visualization code—it emerges directly from the field definitions in Eqs.~\ref{eq:E}--\ref{eq:S}.

\subsection{Phase Synchronization}

Both $\mathbf{E}$ and $\mathbf{B}$ rotate in phase with argument $kz$, as evident from the helical trajectories remaining in sync along the $z$-axis. This synchronization reflects the CM field matrix structure where $f = \mathbf{E} + ic\mathbf{B}$ treats electric and magnetic components as real and imaginary parts of a unified complex quantity.

\subsection{Energy Flux}

The Poynting vector $\mathbf{S}$ consistently points in the $+z$ direction at all sampled positions, visualizing the unidirectional energy flow characteristic of traveling waves. The time-averaged energy flux $\langle S \rangle = \frac{1}{2\mu_0} E_0 B_0$ is represented by the steady orientation of green arrows.

\section{Educational and Research Applications}
\label{sec:applications}

\subsection{Pedagogical Value}

This visualization serves as teaching material for electromagnetic wave theory by:
\begin{enumerate}
\item Making abstract field orthogonality geometrically tangible
\item Clarifying the distinction between field vectors (arrows) and field trajectories (helices)
\item Demonstrating how $\mathbf{E} \times \mathbf{B}$ produces a constant-direction vector from two rotating vectors
\end{enumerate}

The interactive HTML output allows students to rotate and zoom the 3D scene, building intuition for the wave structure.

\subsection{CM Theory Validation}

From the CM perspective, this visualization empirically confirms that:
\begin{itemize}
\item The $\mathfrak{su}(3)$ embedding of electromagnetic fields~\cite{to2026m3c} naturally produces orthogonal vector triplets
\item Phase coherence in $f = \mathbf{E} + ic\mathbf{B}$ manifests as geometric phase locking in real space
\item Energy conservation (Poynting theorem) appears as a direct consequence of the field algebra structure~\cite{to2026unified}
\end{itemize}

\section{Limitations and Future Work}
\label{sec:future}

The current implementation is restricted to:
\begin{itemize}
\item Plane waves in vacuum (no dispersion, absorption, or scattering)
\item Static snapshots (time-frozen at $t=0$)
\item Linear polarization visualization
\end{itemize}

Future extensions could include:
\begin{enumerate}
\item Time-evolution animations showing $\mathbf{E}(z,t)$ and $\mathbf{B}(z,t)$ oscillations
\item Material media with $\epsilon_r \neq 1$, $\mu_r \neq 1$
\item Superposition of multiple waves (interference, standing waves)
\item Integration with CM operator visualization (showing $\nabla \cdot$ action on $\mathbf{F}$)
\end{enumerate}

\section{Conclusion}
\label{sec:conclusion}

We have presented an open-source visualization of electromagnetic wave structure derived from the CM-Maxwell unified equation. The code demonstrates that fundamental electromagnetic properties—field orthogonality, phase synchronization, and directional energy flux—emerge naturally from the $\mathfrak{su}(3)$-valued field representation without additional postulates. This work complements the theoretical formalism in~\cite{to2026unified} by providing empirical visual evidence and pedagogical tools for understanding the geometric structure of classical electromagnetism.

\section*{Code Availability}

The Python implementation is available at:

\begin{center}
\url{https://github.com/TdotOdot/cm-maxwell-visualization}
\end{center}

Dependencies: \texttt{numpy} $\geq 1.24.0$, \texttt{plotly} $\geq 5.14.0$. The repository will be archived on Zenodo upon publication with an assigned DOI.


\printbibliography

\end{document}